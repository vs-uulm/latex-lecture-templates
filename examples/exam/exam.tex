\documentclass{../../uulm-exam}
%\documentclass[milogo]{../../uulm-exam}

\usepackage{algorithmic}
\usepackage{tabularx}
\usepackage{pdfpages}
\usepackage{subfigure}
\usepackage{hyphenat}
\usepackage{wrapfig}
\usepackage[weather]{ifsym}
\usepackage{wasysym}

\lstset{language=Java,numbers=none}

\setboolean{showsolutions}{false}
\setboolean{showsolutions}{true}

% Feld für Codewort für den Aushang zeigen
\setboolean{showcodeword}{false}
% Feld für Notenbonus
\setboolean{showbonus}{true}
% Zeige Felder für Korrekturstatus in Unteraufgaben-Granularität, füge Felder für Korrektur-Zeichen pro Unteraufgabe hinzu
\setboolean{showsubtaskboxes}{true}
% Zeige Barcodes auf jeder Seite
\setboolean{showbarcodes}{false}
% Zeige Einträge für Subtasks
\setboolean{showsubtaskboxes}{true}

\identifier{UIST 2013}

\begin{document}
\mysecret{20} % Geheime Zahl, die Barcodes generiert um "gefaelschte" Blaetter zu verhindern - sollte je Klausur unterschiedlich sein
\title{User Interface Softwaretechnologie}
\date{08.07.2013}
\institute{Institut für Medieninformatik}
\duration{90 Minuten}
\examiner{Prof. Michael Weber}


\begin{hints}
\begin{hintslist}
\item Bitte prüfen Sie die Vollständigkeit Ihrer Aufgabenblätter
(insgesamt \gettotalassignments{} Aufgaben auf \pageref{LastPage} Seiten)!
\item Lösungen bitte nur auf die Aufgabenblätter und nicht mit Rot- oder
Bleistift schreiben!
\item Als Schmierzettel können die Rückseiten verwendet werden.
Sollten sich Teile Ihrer Lösung nicht direkt bei der Aufgabe befinden, so
kennzeichnen Sie dies bitte deutlich bei der jeweiligen Aufgabe!
\end{hintslist}
\end{hints}
%% MI Hinweisliste
% \begin{hints}
% \begin{hintslist}
% \item Bitte prüfen Sie die Vollständigkeit Ihrer Aufgabenblätter
% (insg. \gettotalassignments{} Aufgaben auf \pageref{LastPage} Seiten)!
% \item Lösungen bitte nur auf die Aufgabenblätter und nicht mit Rot- oder
% Bleistift schreiben!
% \item Falls Ihnen der Platz auf einem Aufgabenblatt (inkl. Rückseite) nicht ausreicht, verwenden Sie bitte die hinten angehängten Extrablätter.
% \item Sollten sich Teile Ihrer Lösung nicht direkt bei der Aufgabe befinden, so
% kennzeichnen Sie dies bitte deutlich bei der jeweiligen Aufgabe!
% \item Herausgenommene Blätter beschriften Sie bitte zusätzlich mit ihrem Namen und Matrikelnummer.
% \end{hintslist}
% \end{hints}

\begin{tools}
Ein beidseitig handbeschriebenes DIN-A4-Blatt.
\end{tools}

\maketitle

% einbinden der einzelnen Aufgaben und Lösungen
\begin{assignments}
\assignment{MVC}
	
\begin{enumerate}
\item Zeichnen Sie das modalitätsunabhängige Modell der Interaktion: Beschriften
Sie alle Komponenten und deren Verbindungen. Beschreiben Sie kurz den
Interaktionsprozess.
\subpoints{4}

\item Was beinhalten die Komponenten aus a) im konkreten Fall, wenn es sich bei
dem zu Grunde liegenden System um einen einfachen MP3-Player handelt?
\subpoints{4}

\end{enumerate}

% Jede Hauptaufgabe hat eine Überschrift und die Gesamtpunkte
\assignment{Fenstersysteme und Präsentation}

% Teilaufgabe als enumerate items
\begin{enumerate}

\item Bei einer Smartphone-Benutzeroberfläche kann immer nur eine Applikation
gleichzeitig angezeigt werden. Welche Komponente eines Fenstersystems --
verglichen mit denen eines normalen PC-Systems -- kann dadurch erheblich vereinfacht
werden? Begründen Sie Ihre Antwort.
\subpoints{2} % Die Punkte für die Teilaufgabe

% Zu jeder Teilaufgabe ein solution-Block der form \begin{solution}{Platz für Lösung}[Linien true/false] ... \end{solution}
\begin{solution}{5}[squared]
Fenstermanager. Fenster können sich nicht mehr überlappen,
Fokus/Eventzustellung/damage/redraw entfällt 
\end{solution}

\item Mit welcher Methode kann die Bildtiefe und damit der Speicherbedarf des
Framebuffers reduziert werden, ohne die Anzahl an darstellbaren Farben
einzuschränken? Skizzieren Sie die dafür nötigen Speicherstrukturen. Bei welcher
Applikation eines modernen Smartphones würde die Methode nicht angewendet werden
können und warum nicht?
\subpoints{3}

\begin{solution}{3}[blank]
Farbtabelle (Skript Teil 2, Folie 47)
\end{solution}

\item Ein \wordline[Lückentext]{8em} kann sehr einfach \wordline[korrigiert]{6em} werden und ist geeignet auch \wordline[komplexe]{4em} Sachverhalte abzufragen.

\end{enumerate}

\end{assignments}

% Zusatzblätter
\newpage
\makeemptysheet
\makeemptysheet
\makeemptysheet

% JavaDoc
%\includepdf,angle=90,nup=1x2]{API-Aufgabe-3}
% \includepdf[pages={4,1,3,2},angle=90,nup=1x2]{coffee-javadoc}

\end{document}
