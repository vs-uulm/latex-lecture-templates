\documentclass{klausur}

\usepackage{algorithmic}
\usepackage{tabularx}
\usepackage{pdfpages}
\usepackage{subfigure}
\usepackage{hyphenat}
\usepackage{wrapfig}
\usepackage[weather]{ifsym}
\usepackage{wasysym}

\lstset{language=Java,numbers=none}

\setboolean{showsolutions}{false}
% \setboolean{showsolutions}{true}

\begin{document}
\title{User Interface Softwaretechnologie}
\date{04.04.2012}
\institute{Institut für Medieninformatik}
\duration{90 Minuten}
\examiner{Prof. Michael Weber}
\maketitle

% Außer die Titelseite soll überall rechts mehr Rand sein
\addtolength{\hoffset}{-1cm}

% einbinden der einzelnen Aufgaben und Lösungen
\inputassignment{01-interaktionsmodell}{01-interaktionsmodell-lsg}
\inputassignment{02-praesentation}{02-praesentation-lsg}

% notwendig, damit die Punkte der letzten Aufgabe korrekt angezeigt werden
\refstepcounter{subpoints}\label{pointsref@\theassignment}%

% Zusatzblätter
\identifier{UIST 2007}
\makeemptysheet
\makeemptysheet
\makeemptysheet

% JavaDoc
%\includepdf,angle=90,nup=1x2]{API-Aufgabe-3}
% \includepdf[pages={4,1,3,2},angle=90,nup=1x2]{coffee-javadoc}

\end{document}