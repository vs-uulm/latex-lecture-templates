% Jede Hauptaufgabe hat eine Überschrift und die Gesamtpunkte
\assignment{MVC}
	
% Teilaufgabe als enumerate items
\begin{enumerate}
\item Zeichnen Sie das modalitätsunabhängige Modell der Interaktion: Beschriften
Sie alle Komponenten und deren Verbindungen. Beschreiben Sie kurz den
Interaktionsprozess.
% Die Punkte für die Teilaufgabe
\subpoints{4}

% Zu jeder Teilaufgabe ein solution-Block der form \begin{solution}{Platz für Lösung}[Linien true/false] ... \end{solution}
\begin{solution}{3}[lines]
MVC-Skizze aus dem Skript, jeweils .5 Punkte pro Eintrag.
\end{solution}

\item Was beinhalten die Komponenten aus a) im konkreten Fall, wenn es sich bei
dem zu Grunde liegenden System um einen einfachen MP3-Player handelt?
\subpoints{4}

\begin{solution}{6}[blank]
\begin{description} 
\item[User Modell] Stop, Play, vorwärts, rückwärts, An, aus
\item[Model] z.b. internetstreaming, fileposition, dateiname, länge, codec,
bitrate
\item[View] Statusbar, titel, Tags, Albumcover aus dem Internet 
\item[Controller] buttons, on/off
\end{description}
\end{solution}


\item Eine beispielhafte Multiple Choice Aussage (je Aussage: 1 Punkt bei korrekter Antwort, 1 Punkt Abzug bei inkorrekter Antwort, 0 Punkte bei fehlender Antwort; mind. 0 Punkte für die gesamte Teilaufgabe).
\subpoints{2}

\begin{multiplechoice}
\choice{wahr}{Lorem ipsum dolor sit amet, consetetur sadipscing elitr, sed diam nonumy eirmod tempor invidunt ut labore et dolore magna aliquyam erat, sed diam voluptua.}
\choice{wahr}{Das ist eine wahre Aussage}
\choice{falsch}{$1+1 = 3$}
\end{multiplechoice}

\end{enumerate}
